% "Тема 9. ИСР"

\documentclass[a4paper,12pt]{article} % тип документа

% report, book

%  Русский язык

\usepackage[T2A]{fontenc}			% кодировка
\usepackage[utf8]{inputenc}			% кодировка исходного текста
\usepackage[english,russian]{babel}	% локализация и переносы


% Математика
\usepackage{amsmath,amsfonts,amssymb,amsthm,mathtools} 


\usepackage{wasysym}

%Заговолок
\author{Косоруков Роман, ИВТ, 3 курс, 1 подгруппа}
\title{Тема 9. ИСР}
\date{\today}

\begin{document}

\maketitle
\newpage
\section{Таблица интегралов и дифференциалов}
\Large
\begin{tabular}{| l | l |}
\hline
\textbf{Интегралы} & \textbf{Дифференциалы} \\
\hline
$\int 0 \cdot dx = C$ & $d(c) = 0, c = const$\\[0.5cm]
$\int dx = \int 1 \cdot dx = x + C $ & $d(x^n) = nx^{n-1}dx$\\[0.5cm]
$\int x^n \cdot dx = \frac{x^{n+1}}{n+1} + C$ & $d(a^x) = a^x \cdot ln adx$\\
$n \neq -1, x > 0$ & \\[0.5cm]
$\int \frac{dx}{x} = ln|x| + C$ & $d(e^x) = e^xdx$\\[0.5cm]
$\int a^x dx = \frac{a^x}{ln a} + C $ & $d(log_ax) = \frac{dx}{x ln a} $\\[0.5cm]
$\int a^x dx = e^x + C $ & $d(ln x) = \frac{dx}{x} $\\[0.5cm]
$\int sinxdx = -cosx + C $ & $d(sinx) = cosxdx $\\[0.5cm]
$\int cosxdx = sinx + C $ & $d(cosx) = -sinxdx $\\[0.5cm]
\hline
\end{tabular}

\end{document}